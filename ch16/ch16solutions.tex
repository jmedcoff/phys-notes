\documentclass{article}

\usepackage{amsmath}
\usepackage{geometry}
\geometry{a4paper}

\newcommand{\uveci}{{\boldsymbol{\hat{\textnormal{\bfseries\i}}}}}
\newcommand{\uvecj}{{\boldsymbol{\hat{\textnormal{\bfseries\j}}}}}
\newcommand{\uveck}{{\boldsymbol{\hat{\textnormal{\bfseries{k}}}}}}

\title{Chapter 16 Practice Problems}
\author{Jason Medcoff}
\date{}

\begin{document}
	
	\maketitle
	
	\section{Preliminaries}
	In this chapter we learn about measuring electric potential in the context of point charges and collections of charges. We apply the idea of electric fields as a way to store energy when we talk about capacitors.
	
	\section{Problems}
	
	\subsection{Problem 1}
	
	An electron moving at $3.0 \times 10^5 m/s$ is stopped by a conductor. What is the minimum potential difference the conductor must have? 
	
	\textit{Solution.}
	
	For a particle in motion to completely stop, its kinetic energy must be reduced to zero. Because of conservation of energy, we know that an energy equal to the particle's KE must be applied from something else. Thus, the potential energy of the conductor must be equal to the initial KE of the particle.
	
	So we have that
	
	\begin{equation*}
	\begin{split}
	KE_{electron} & = PE_{conductor} \\
	\frac{1}{2}mv^2 & = -e \Delta V \\
	\frac{-mv^2}{2e} & = \Delta V \\
	\frac{(-9.11 \times 10^{-31} kg)(3 \times 10^5 m/s)^2}{2(1.602 \times 10^{-19} C)} & = \Delta V \\
	-0.256 V & = \Delta V
	\end{split}
	\end{equation*}
	
	\subsection{Problem 2}
	
	An electron moving at velocity $v$, moves between two charged plates producing an electric field of $2.5 \times 10^4 N/C$. 
	
	
	
	
	
	
	
	
	
\end{document}