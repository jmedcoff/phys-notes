\documentclass{article}

\usepackage{amsmath}
\usepackage{geometry}
\geometry{a4paper}

\newcommand{\uveci}{{\boldsymbol{\hat{\textnormal{\bfseries\i}}}}}
\newcommand{\uvecj}{{\boldsymbol{\hat{\textnormal{\bfseries\j}}}}}
\newcommand{\uveck}{{\boldsymbol{\hat{\textnormal{\bfseries{k}}}}}}

\title{Chapter 16 Practice Problems}
\author{Jason Medcoff}
\date{}

\begin{document}
	
	\maketitle
	
	\section{Preliminaries}
	In this chapter we learn about measuring electric potential in the context of point charges and collections of charges. We apply the idea of electric fields as a way to store energy when we talk about capacitors.
	
	\section{Problems}
	
	\subsection{Problem 1}
	
	An electron moving at $3.0 \times 10^5 m/s$ is stopped by a conductor. What is the minimum potential difference the conductor must have? 
	\newline \newline \newline \newline \newline 
	\newline \newline \newline \newline \newline 
	
	\subsection{Problem 2}
	
	An electron moving at velocity $v$, moves along the perpendicular between two charged plates producing an electric field of $2.5 \times 10^4 N/C$. Its final velocity is $3v$ after it has moved 2cm. What are its final velocity, initial velocity, and acceleration?
	\newline \newline \newline \newline \newline 
	\newline \newline \newline \newline \newline 
		
		
	\subsection{Problem 3}
	
	Two point charges are located in the Cartesian plane. A 9nC charge lies at (0,0) and a 3nC charge lies at (3,4). What is the electric potential at (5,1)?
	
	
	
	
	
\end{document}