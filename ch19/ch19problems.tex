\documentclass{article}

\usepackage{amsmath}
\usepackage{geometry}
\geometry{a4paper}

\newcommand{\uveci}{{\boldsymbol{\hat{\textnormal{\bfseries\i}}}}}
\newcommand{\uvecj}{{\boldsymbol{\hat{\textnormal{\bfseries\j}}}}}
\newcommand{\uveck}{{\boldsymbol{\hat{\textnormal{\bfseries{k}}}}}}

\title{Chapter 19 Practice Problems}
\author{Jason Medcoff}
\date{}

\begin{document}
	\maketitle
	
	\section{Preliminaries}
	
	In this chapter we are introduced to magnetic forces and behavior of charge in the presence of magnetic fields. We assume that magnetic fields exist, but next chapter we will show where they come from.
	
	\section{Problems}
	
	\subsection{Problem 1}
	
	A 60 cm long wire is oriented from east to west above the equator and carries an eastward current of 5 amps. What is the magnitude and direction of the force on the wire due to earth's 30 $\mu$T northward magnetic field?
	
	$\newline \newline \newline \newline$
	$\newline \newline \newline \newline$
	$\newline \newline \newline \newline$
	
	\subsection{Problem 2}
	Find the magnetic field a distance of 10 cm from a wire carrying 3 A of current.
	
	$\newline \newline \newline \newline$
	$\newline \newline \newline \newline$
	$\newline \newline \newline \newline$
	
	\subsection{Problem 3}
	Two wires \textit{A} and \textit{B} experience magnetic force when placed in two different magnetic fields $B_A$ and $B_B$. Wire \textit{A} is twice as long and carries three times as much current as wire \textit{B}. Wire \textit{B} experiences seven times as much force as wire \textit{A}. How much stronger is $B_A$ than $B_B$?
	
	
	
	
	
\end{document}