\documentclass{article}

\usepackage{amsmath}
\usepackage{geometry}
\geometry{a4paper}

\newcommand{\uveci}{{\boldsymbol{\hat{\textnormal{\bfseries\i}}}}}
\newcommand{\uvecj}{{\boldsymbol{\hat{\textnormal{\bfseries\j}}}}}
\newcommand{\uveck}{{\boldsymbol{\hat{\textnormal{\bfseries{k}}}}}}

\title{Chapter 27 Practice Problems}
\author{Jason Medcoff}
\date{}

\begin{document}
	\maketitle
	
	\subsection{Problem 1}
	What daughter nucleus would be formed if $^{238}_{92} U$ could undergo $\alpha$ decay? $\beta^-$ decay? $\gamma$ decay?
	
	\textit{Solution.}
	We know that alpha decay emits a helium nucleus: two protons and two neutrons. So, $238 - 4$ for mass numbers yields 234, and $92 - 2$ for atomic numbers yields 90. Therefore, we have a daughter of $^{234}_{90} Th$.
	
	The reaction for beta decay emits one electron. So, to conserve charge, the daughter must have charge of $92 - (-1)$ gives 93, so the daughter is $^{238}_{93} Np$.
	
	Gamma decay emits no mass or charge, so the daughter nucleus is $^{238}_{92} U$.
	
	
	\subsection{Problem 2}
	Radioactive isotopes of strontium are known to be dangerous to animals, since it chemically behaves like calcium and is taken up by the body and deposited in bone. Much of the radioactive strontium in the atmosphere is Sr-90 which was mostly released by atmospheric nuclear bomb testing in the 20th century. Assume no new strontium has been introduced to the atmosphere for 28 years. What percentage of the original nuclei remain in the atmosphere? Strontium has a half life of 28.5 years.
	
	\textit{Solution.}
	In any given volume of air we have
	$$ N = N_0 e^{-\lambda t} $$
	radioactive nuclei left. The percentage left is
	$$ \frac{N}{N_0} = e^{-\lambda t} . $$
	The decay constant can be found as
	$$ \lambda = (0.693)/28.5 \ y = 0.02432 \ y^{-1} . $$
	Then the percentage left is
	$$ e^{(-0.02432)(28)} = 0.506 $$
	or 50.6 percent.
	
	
	\subsection{Problem 3}
	The shroud of Turin was originally believed to be about two thousand years old, but was determined by radioactive dating to be about 700 years old. Assuming the shroud had a carbon-14 activity of 0.23 Bq per gram when it was made, what is its activity now?
	
	\textit{Solution.}
	The number of nuclei left in the shroud is given by
	$$ N = N_0 e^{-\lambda t} $$
	and the activity $R = \lambda N$ is
	$$ R = R_0 e^{-\lambda t} $$
	The decay constant for carbon-14 is
	$$ (0.693)/(5730) = 1.21 \times 10^{-4} $$
	and so, the per gram activity is
	$$ R = (0.23) e^{-(1.21\times 10^{-4})(700)} = 0.21 \ Bq . $$
	
	
	
	
	
	
	
	
\end{document}