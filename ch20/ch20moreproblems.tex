\documentclass{article}

\usepackage{amsmath}
\usepackage{geometry}
\geometry{a4paper}

\newcommand{\uveci}{{\boldsymbol{\hat{\textnormal{\bfseries\i}}}}}
\newcommand{\uvecj}{{\boldsymbol{\hat{\textnormal{\bfseries\j}}}}}
\newcommand{\uveck}{{\boldsymbol{\hat{\textnormal{\bfseries{k}}}}}}

\title{Chapter 19 Practice Problems}
\author{Jason Medcoff}
\date{}

\begin{document}
	\maketitle
	
	\subsection{Problem 1}
	A cube with side length of 0.5 m is placed in a uniform magnetic field of strength 0.45 T oriented in the direction of a diagonal of the front face of the cube. Find the magnetic flux through each of the faces of the cube. What is the total flux through the cube?
	
	$\newline \newline \newline \newline \newline \newline \newline \newline \newline \newline \newline \newline$
	
	\subsection{Problem 2}
	A metal bar is moving perpendicular to a 1.5 T magnetic field with constant velocity 150 m/s. The bar delivers a current of 5.5 A to a 6.0 $\Omega$ load. How long is the bar, and how much force is necessary to keep the bar moving at its velocity?
	
		$\newline \newline \newline \newline \newline \newline \newline \newline \newline \newline \newline \newline$
	
	\subsection{Problem 3}
	\textbf{Challenge Problem:} A ring of aluminum is placed over the iron core of an electromagnet. The current in the coil is 60 Hz alternating current. Use Lenz's Law to show that the flux change due only to the time varying magnetic field through the ring is insufficient to explain why the ring is repelled from the electromagnet.
	
	
	
	
	
	
	
	
	
	
	
\end{document}