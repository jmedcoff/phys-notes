\documentclass{article}

\usepackage{amsmath}
\usepackage{geometry}
\geometry{a4paper}

\newcommand{\uveci}{{\boldsymbol{\hat{\textnormal{\bfseries\i}}}}}
\newcommand{\uvecj}{{\boldsymbol{\hat{\textnormal{\bfseries\j}}}}}
\newcommand{\uveck}{{\boldsymbol{\hat{\textnormal{\bfseries{k}}}}}}

\title{Chapter 15 Practice Problems}
\author{Jason Medcoff}
\date{}

\begin{document}
	
	\maketitle
	
	\section{Preliminaries}
	
	We have already learned about electric charge and how it behaves. Namely, opposite charges attract and like charges repel. Coulomb's Law governs this behavior. We can generalize to a single charge in space, and determine the electric field surrounding this charge.
	When we combine the ideas of electric field and closed surfaces, we can talk about electric flux, which allows us to determine the amount of charge within a body in a simpler way. We can also apply Gauss's Law to exploit certain geometric symmetries, and make solving physics problems much simpler.
	
	\section{Problems}
	
	\subsection{Problem 1}
	
	Two electrons and a proton lie on the $y$ axis; the electrons are 10 and 15 cm from the origin, and the proton is 25 cm from the origin. What is the electric force on the proton?
	\newline \newline \newline \newline \newline 
    \newline \newline \newline \newline \newline 
	\subsection{Problem 2}
	
	A metal ball of mass $m$ and charge $-2q$ is levitated above a conductor at a height $h$ by an electric force. What charge must be on the conductor in order to hold the ball steady?
    \newline \newline \newline \newline \newline 
    \newline \newline \newline \newline \newline 
	
	\subsection{Problem 3}
	
	Consider a cylinder with radius 2 cm. If an electric field of $2.3 \times 10^5 \ N/C$ permeates the bottom of the cylinder with an angle of 30 degrees to the bottom, what is the flux through this part of the cylinder?
	
	
	
	
	
	
	
	
	
	
	
\end{document}