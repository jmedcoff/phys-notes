\documentclass{article}

\usepackage{amsmath}
\usepackage{geometry}
\geometry{a4paper}

\newcommand{\uveci}{{\boldsymbol{\hat{\textnormal{\bfseries\i}}}}}
\newcommand{\uvecj}{{\boldsymbol{\hat{\textnormal{\bfseries\j}}}}}
\newcommand{\uveck}{{\boldsymbol{\hat{\textnormal{\bfseries{k}}}}}}

\title{Chapter 24 Practice Problems}
\author{Jason Medcoff}
\date{}

\begin{document}
	\maketitle
	
	\subsection{Problem 1}
	A helium neon laser emits light at wavelength 633 nm incident on two narrow slits. The third bright fringe of the resulting pattern is 15 degrees from the central maximum. What distance separates the slits?
	
	\textit{Solution.}
	We know that we have an equation governing the distribution of bright spots resulting from an interference pattern. Namely,
	$$ d\sin\theta = m\lambda $$
	and we can rearrange to solve for $d$.
	\begin{equation*}
	\begin{split}
	d & = \frac{m\lambda}{\sin\theta} \\
	  & = \frac{3(633 \times 10^{-9})}{\sin 15} \\
	  & = 7.3 \times 10^{-6} m .
	\end{split}
	\end{equation*}
	
	
	\subsection{Problem 2}
	We have a double slit arrangement with a distance between the slits of 0.1 mm and light of wavelength 455 nm. How many dark spots in total will appear given a very large screen?
	
	\textit{Solution.}
	Here we will simply assume that the screen extends forever. Then all fringes from 0 to 90 degrees can fall on the screen. If the last fringe falls on the screen at 90 degrees, we can find its order as
	$$ m = \frac{d}{\lambda}\sin\theta - \frac{1}{2} . $$
	Plugging in known values, we obtain
	$$ m = \frac{0.1 \times 10^{-3}}{455 \times 10^{-9}} \sin 90 - \frac{1}{2} = 219 . $$
	There is a dark fringe for every $m$ from 0 to 219, so we have 220 in total. Due to the symmetry of double slit interference on the screen, we multiply by two for the total, obtaining 440 dark spots.
	
	
	\subsection{Problem 3}
	A physics student would like to measure the diameter of a strand of her hair. She places the hair between two flat glass plates, and illuminates the plates with light of wavelength 552 nm. Counting the number of bright fringes, she observes 125 bright spots. What is the diameter of the hair?
	
	\textit{Solution.}
	We know that because the plates of glass are assumed thin, we can use the equations describing thin films. Namely, the reflections from the boundaries here will cause a net 180 degree phase shift. The condition thus for bright fringes is
	$$ 2nt = (m+\frac{1}{2})\lambda $$
	and since we have $m = 124$ (the central fringe is of order zero), we can show that
	$$ t = \frac{(m + 1/2)\lambda}{2n} = \frac{124.5(552\times 10^{-9})}{2(1)} = 3.44 \times 10^{-5} m . $$
















\end{document}