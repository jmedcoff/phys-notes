\documentclass{article}

\usepackage{amsmath}
\usepackage{geometry}
\geometry{a4paper}

\newcommand{\uveci}{{\boldsymbol{\hat{\textnormal{\bfseries\i}}}}}
\newcommand{\uvecj}{{\boldsymbol{\hat{\textnormal{\bfseries\j}}}}}
\newcommand{\uveck}{{\boldsymbol{\hat{\textnormal{\bfseries{k}}}}}}

\title{Chapter 24 Practice Problems}
\author{Jason Medcoff}
\date{}

\begin{document}
	\maketitle
	
	\subsection{Problem 1}
	A helium neon laser emits light at wavelength 633 nm incident on two narrow slits. The third bright fringe of the resulting pattern is 15 degrees from the central maximum. What distance separates the slits?
	
	$ \newline \newline \newline \newline \newline \newline \newline \newline \newline \newline \newline \newline $
	
	\subsection{Problem 2}
	We have a double slit arrangement with a distance between the slits of 0.1 mm and light of wavelength 455 nm. How many dark spots in total will appear given a very large screen?
	
	$ \newline \newline \newline \newline \newline \newline \newline \newline \newline \newline \newline \newline \newline \newline$
	
	
	\subsection{Problem 3}
	A physics student would like to measure the diameter of a strand of her hair. She places the hair between two flat glass plates, and illuminates the plates with light of wavelength 552 nm. Counting the number of bright fringes, she observes 125 bright spots. What is the diameter of the hair?
	
	$ \newline \newline \newline \newline \newline \newline \newline \newline \newline \newline $
















\end{document}